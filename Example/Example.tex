
\documentclass[a4paper]{scrartcl}


\usepackage[T1]{fontenc}
\usepackage[utf8]{inputenc}
\usepackage{graphicx}
\usepackage[automark]{scrpage2}
\usepackage{geometry}
\usepackage{setspace}
\onehalfspacing 



\begin{document}


\subsection{The idea}
New trends within internet technology have led to the development of communities and new concepts of hosted services. Social-networking sites, content sharing sites, Web logs, and tagging communities have enabled people to easily become publishers and consumers at almost no cost. They produce and perceive information on a global scale. This creates interesting opportunities; however, a range of new problems arises.
When reading and consuming information on the Web 2.0, with every individual having the possibility to publish content, the issue of information overload becomes eminent. How can subscriptions to several hundred blogs be managed and the desired information still be extracted from them?
When publishing to the web, to build up a reputation may become very difficult. The originator has no possibility to address solely those readers to whom it is relevant. How can a publisher define himself and stand out from the crowd of personal channels?

\subsection{Automated filtering} 

The reader may realize that incoming information from this Software only comes  through subscriptions. On the one hand, this allows having effective control over the received feeds. Spam becomes impossible. On the other hand, information can only spread in the network via redistribution, which is done completely manually until now. This can result in slow information diffusion through the network of nodes.
To avoid this some automated redistribution has to be invented. Optimally, this function should redistribute those feeds, which the user would have shared with a high probability anyway. To capture the user's criterion for a valuable feed, some metric, a rating for a feed's value, has to be applied. The simplest approach seems to be a rating by a single numerical variable. This variable should describe the individual's trust towards the source of the information regarding the topic of the feed. Assuming possible values from 0.0 to 1.0, where 0.0 represents no trust at all and 1.0 adjudges absolute competency in a certain subject, the calculation of the automated-redistribution metric boils down to simple multiplications along the edges of the social graph. In other words, each post will be tagged with its rating on redistribution. Starting with a rating of 1.0 at the author's node, a post will be received by a subscriber and the rating will be updated by multiplication of the original value and the receiver's rating. Redistribution will be continued this way until the rating drops beneath 0.5.

\end{document}