% -----------------------------------------------------------------------------
% LaTeX-Template für die Bachelor-Thesis
%
% Autor:	Markus Ringel <markus.ringel@gmail.com>
% Version:	0.1 (2009-04-22)
% 
% Das Template verwendet die KOMA-Script Dokumentenklassen und setzt einige
% LaTeX Pakete vorraus, die unter Umständen nicht im Standardumfang der loka-
% len LaTeX Distribution enthalten sind. Erhältlich sind diese Pakete über das
% CTAN (http://www.ctan.org/). Zur Installation siehe die Dokumentation des
% jeweiligen Pakets und der LaTeX Distribution.
%
% Weiterhin ist das Template auf die Verwendung von UTF-8 als Zeichenkodierung
% ausgelegt. Dadurch können Umlaute und andere Sonderzeichen direkt verwendet
% werden.
%
% Es sind für die variablen Felder auf der Titelseite und den formalen Erklär-
% ungen Befehle definiert, mit denen diese gesetzt werden können. Siehe dazu
% weiter unten den Abschnitt 'Vorspann'.
%
% Zum Übersetzen sind in der Regel folgende Durchläufe nötig:
%  1. latex arbeit
%  2. bibtex arbeit (nur wenn schon Zitate im Text vorhanden sind)
%  3. makeglossaries arbeit
%  4. latex arbeit
%  5. latex arbeit (damit Referenzen und Verzeichnisse stimmen)
% -----------------------------------------------------------------------------
\documentclass[ngerman,a4paper,11pt,twoside]{scrreprt}

% Erforderliche Pakete
\usepackage[utf8]{inputenc}
\usepackage{babel}
\usepackage{graphicx}
\usepackage{fancyhdr}
\usepackage{multirow}
\usepackage{ifthen}
\usepackage{tabularx}

% Laden von weiteren, nützlichen Paketen (je nach Bedarf)
%\usepackage{subfigure} 
%\usepackage{varioref}
%\usepackage{framed}
%\usepackage{url}
% ...

% Pakete für Verzeichnisse. Diese sind hier aufgeführt, da mindestens das
% 'glossaries' Paket nach z.B. hyperref geladen werden muss.
\usepackage{bibgerm}
\usepackage[toc,acronym]{glossaries} % Nachfolger von 'glossary', Doku siehe
% http://tug.ctan.org/tex-archive/macros/latex/contrib/glossaries/glossaries-manual.html

% Glossar(e) laden und erstellen
\makeglossaries
\loadglsentries{glossar} % Name der Glossardatei (ohne .tex)

% Befehlsdefinitionen
%
% Befehlsdefinitionen für das LaTeX Template
%
% Autor:	Simon Lehmann <simon.lehmann@gmx.de>
% Version:	1.0 (2008-06-19)
%

% Sorgt dafür, dass Kapitel immer auf einer rechten Seite beginnen.
\newcommand{\clearemptydoublepage}{\newpage{\pagestyle{empty}\cleardoublepage}}

% Befehle zum Setzen und Abrufen des Ortsnamen, der in der Erklärung verwendet wird
\newsavebox{\citybox}
\newcommand{\city}[1]{\sbox{\citybox}{#1}}
\newcommand{\thecity}{\usebox{\citybox}}

% Befehle zum Setzen und Abrufen des Referenten
\newsavebox{\supervisorbox}
\newcommand{\supervisor}[1]{\sbox{\supervisorbox}{#1}}
\newcommand{\thesupervisor}{\usebox{\supervisorbox}}

% Befehle zum Setzen und Abrufen des Korreferenten
\newsavebox{\cosupervisorbox}
\newcommand{\cosupervisor}[1]{\sbox{\cosupervisorbox}{#1}}
\newcommand{\thecosupervisor}{\usebox{\cosupervisorbox}}

\makeatletter % Zugriff auf interne Befehle
% Autorenname, der mit \author{...} gesetzt wurde
\newcommand{\theauthor}{\@author}

% Datum, das mit \date{...} gesetzt wurde
\newcommand{\thedate}{\@date}

% Titel der Arbeit, der mit \title{...} gesetzt wurde
\newcommand{\thetitle}{\@title}
\makeatother

% Flags für Verbreitungsformen
\newboolean{publishlibrary}
\setboolean{publishlibrary}{false}
\newcommand{\publishlibrary}{\setboolean{publishlibrary}{true}}

\newboolean{publishtitle}
\setboolean{publishtitle}{false}
\newcommand{\publishtitle}{\setboolean{publishtitle}{true}}

\newboolean{publishdocument}
\setboolean{publishdocument}{false}
\newcommand{\publishdocument}{\setboolean{publishdocument}{true}}


% Globale Formatierungseinstellungen
\setlength{\parindent}{0pt}
\setlength{\parskip}{6pt}
\renewcommand{\encodingdefault}{OT1}
\renewcommand{\familydefault}{cmss} % Schriftfamilie auf Sans Serif
\renewcommand{\glsdisplayfirst}[4]{\textit{#1#4}}

% Wenn größerer Zeilenabstand gewünscht ist, je nach Schriftgröße:
% Abstand        10pt    11pt    12pt
% -----------------------------------
% anderthalb     1.25    1.21    1.24
% doppelt        1.67    1.62    1.66
%
%\renewcommand{\baselinestretch}{1.21}

% Kopf- und Fußzeilen einrichten
\pagestyle{fancyplain}
\addtolength{\headwidth}{\marginparwidth}
\renewcommand{\chaptermark}[1]{\markboth{#1}{}}
\renewcommand{\sectionmark}[1]{\markright{\thesection\ #1}}
\lhead[\fancyplain{}{\bfseries\thepage}]%
	{\fancyplain{}{\bfseries\rightmark}}
\rhead[\fancyplain{}{\bfseries\leftmark}]%
	{\fancyplain{}{\bfseries\thepage}}
\cfoot{}

\begin{document}
% --------------------------------- Vorspann ----------------------------------
% In den folgenden Zeilen werden die wichtigsten Informationen zur Arbeit ge-
% setzt, die dann im Dokument an den entsprechenden Stellen eingefügt werden.
\title{Bachelor Thesis}
\author{Markus Ringel}
\supervisor{Name des Referenten}
\cosupervisor{Name des Korreferenten}
\date{Datum der Abgabe}
\city{Zürich}

% Folgende Befehle steuern, welchen Verbreitungsformen zugestimmt wird. Nicht
% gewünschte Verbreitungsformen entsprechend auskommentieren.
\publishlibrary  % Einstellung der Arbeit in die Bibliothek der FHW
\publishtitle    % Veröffentlichung des Titels der Arbeit im Internet
\publishdocument % Veröffentlichung der Arbeit im Internet

% Ab hier sollten keine Änderungen des Vorspanns nötig sein!
% -> Weiter zum Hauptteil

\pagenumbering{roman} % Bis zum ersten Kaptiel mit römischen Seitenzahlen

\begin{titlepage}
	\begin{center}
		% Kopf der Seite
		\begin{minipage}{0.5\textwidth}
			\raggedright
			\scalebox{0.65}[0.65]{
			\includegraphics{Logo_HSRM}}
		\end{minipage}
		\hspace{0.3cm}
		\begin{minipage}{0.6\textwidth}
			\raggedright
			{\LARGE Hochschule RheinMain} \\
			{\Large Fachbereich Design Informatik Medien} \\[0.2cm]
		\end{minipage}
		
		\vfill

		{\LARGE Bachelor-Thesis} \\[0.5cm]
		{\large zur Erlangung des akademischen Grades} \\[0.5cm]
		{\Large Bachelor of Science - B.Sc.}
		
		% Titel
		\rule{\textwidth}{1pt}\\[0.5cm]
		{\huge \bfseries A User Interface for Interactive, Computer-Supported Filtering of RSS Feeds}\\[0.4cm]
		\rule{\textwidth}{1pt}
		
		\vfill
		
		% Mitte der Seite
		\begin{tabular}{lr}
			Vorgelegt von Markus Ringel
			am 29.September 2010
			Referent  Ralf Dörner
			Korreferent Dirk Krechel
		\end{tabular}
		
		\vfill
		
		% Fuß der Seite
		
	\end{center}
\end{titlepage}
 % Titelseite (wird automatisch mit Werten von oben gefüllt)
\clearemptydoublepage % Sorgt dafür, dass die nächste Seite rechts beginnt

\section*{Erklärung gemäß ABPO , Ziff. 6.4.3}

Ich versichere, dass ich die Bachelor-Thesis selbstständig verfasst und keine anderen als die angegebenen Hilfsmittel benutzt habe.


\vspace{3cm}
\makebox[.4\linewidth][l]{\thecity, \thedate}\hfill\rule{.4\linewidth}{0.5pt}\\
\makebox[.4\linewidth][l]{}\hfill\makebox[.4\linewidth][l]{\theauthor}

\vfill

\section*{Erklärung zur Verwendung der Bachelor-Thesis}

Hiermit erkläre ich mein Einverständnis mit den im folgenden aufgeführten Verbreitungsformen dieser Bachelor-Thesis:

\newcolumntype{Y}{>{\small\raggedright\arraybackslash}X}
\begin{tabularx}{\linewidth}{|Y|c|c|}
	\hline
	\textbf{Verbreitungsform} & \textbf{Ja} & \textbf{Nein}
	\\\hline
	Einstellung der Arbeit in die Bibliothek der FHW & \ifthenelse{\boolean{publishlibrary}}{$\times$}{} & \ifthenelse{\boolean{publishlibrary}}{}{$\times$}
	\\\hline
	Veröffentlichung des Titels der Arbeit im Internet & \ifthenelse{\boolean{publishtitle}}{$\times$}{} & \ifthenelse{\boolean{publishtitle}}{}{$\times$}
	\\\hline
	Veröffentlichung der Arbeit im Internet & \ifthenelse{\boolean{publishdocument}}{$\times$}{} & \ifthenelse{\boolean{publishdocument}}{}{$\times$}
	\\\hline
\end{tabularx}

\vspace{3cm}
\makebox[.4\linewidth][l]{\thecity, \thedate}\hfill\rule{.4\linewidth}{0.5pt}\\
\makebox[.4\linewidth][l]{}\hfill\makebox[.4\linewidth][l]{\theauthor}
 % Erklärung zur Prüfungsordnung und Verwendung der Arbeit
\clearemptydoublepage

\tableofcontents % Inhaltsverzeichnis generieren
\clearemptydoublepage

% --------------------------------- Hauptteil ---------------------------------
% Im diesen Teil werden die einzelnen Kapitel eingefügt, die sinnvollerweise
% im Verzeichnis 'kapitel' abgelegt werden. Nach jedem Kapitel sollte
% \clearemptydoublepage aufgerufen werden, um das Folgekapitel auf der rechten
% Seite beginnen zu lassen.

\chapter{Einleitung}
\pagenumbering{arabic}

Hier Beginnt das Dokument \ldots
\clearemptydoublepage

%\include{kapitel/zweileitung}
%\clearemptydoublepage

% und so weiter ...

% ---------------------------------- Anhänge ----------------------------------
% In diesem Teil werden alle Anhänge eingefügt, die auch als ganz normale Kapi-
% tel abgelegt werden.
\appendix

%\include{kapitel/einanhang}
%\clearemptydoublepage

% Ausgabe des Glossars (oder der Glossare, wenn mehrere definiert sind)
\printglossaries
\clearemptydoublepage

% Ausgabe des Literaturverzeichnisses
\bibliographystyle{geralpha}
\bibliography{literatur}

\end{document}
